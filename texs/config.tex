\documentclass[
	12pt, % Standard Schriftgröße
	a4paper, % Papierformat
	ngerman, % wird an andere Pakete weitergegeben
	BCOR=10mm, % zusätzlicher Rand auf Innenseite
	DIV=12, % Anzahl Unterteilungen
	parskip=half,  % Halbe Zeile Absatzabstand
	oneside, % einseitiges layout
	%twoside, % zweiseitiges layout
	headsepline, % Linie unter Kopfzeile 
	%footsepline, % Linie über Fußzeile
	headinclude=false, % Kopfzeile nicht in Satzspiegel
	footinclude=false, % Fußzeile nicht in Satzspiegel
	numbers=noenddot, % kein Punkt nach Kapitelnummern
	listof=totoc, % Verzeichnisse im Inhaltverzeichnis
	bibliography=totoc, % Literatur im Inhaltsverzeichnis
%	chapterprefix, % Kapitel ausschreiben
%	appendixprefix, % Anhang ausschreiben
	captions=tableheading, % Tabelle mit Überschrift statt Unterschrift
	usegeometry, % Satzspiegel Berechnung an geometry Paket übertragen
] {scrreprt} % auch möglich, scrartcl oder scrbook

%--------------------------------------
% Metadaten laden, unbedingt anpassen!
%--------------------------------------
%-----------------------------------
% Metadaten zur Arbeit
%-----------------------------------

% Autor
\newcommand{\myAutor}{Max Mustermann}

% Titel der Arbeit
\newcommand{\myTitel}{Latex Vorlage für wissenschaftliche Arbeiten an der THB}

% Untertitel der Arbeit
\newcommand{\myUntertitel}{Mit KOMA-Script und Biber/BibLatex}

% Art der Arbeit
\newcommand{\myThesisArt}{Bachelorarbeit}

% Zu erlangender akademische Grad
\newcommand{\myAkademischerGrad}{Bachelor of Latex (B. Lat.)}

% Betreuer
\newcommand{\myBetreuer}{Prof. Dr. Peter Lustig}

% Zweitgutachter
\newcommand{\myZweitgutachter}{Dipl. Ing. Otto Normalverbraucher}

% Matrikelnummer
\newcommand{\myMatrikelNr}{123456}

% Datum der Abgabe
\newcommand{\myAbgabeDatum}{\today}


%---------------------------------------------
% Momentan nicht benutzte Daten
%---------------------------------------------

% Studiengang
\newcommand{\myStudiengang}{Ingenieurswissenschaften}

% Adresse
\newcommand{\myAdresse}{Magdeburger Straße 50, 14770 Brandenburg an der Havel}

% Ort
\newcommand{\myOrt}{Brandenburg}

% Semesterzahl
\newcommand{\mySemesterZahl}{7}

% Name der Hochschule
\newcommand{\myHochschulName}{Technische Hochschule Brandenburg}

% Firma
\newcommand{\myFirma}{Mustermann GmbH}

%-------------------------------------
% Schriftarten
%-------------------------------------
\usepackage{ifluatex} % Um Fehler zu vermeiden, wenn nicht LuaLaTex verwendet wird
\ifluatex
% LuLaTex wird verwendet
\usepackage{unicode-math}
% Achtung, Schriften müssen auf System installiert sein!
\setmainfont{Times New Roman} % Hauptschrift
\setsansfont{Segoe UI}[Scale=MatchLowercase] % Serifenlose Schrift (Nachfolger Arial)
\setmonofont{Courier New}[Scale=MatchLowercase, FakeStretch=0.85] % Schreibmaschinenschrift
\else
% LuLaTex wird nicht verwendet
\usepackage[T1]{fontenc}
\usepackage{mathptmx}
\usepackage[scaled=0.9]{helvet}
\usepackage{courier}
\usepackage{amssymb} % Mathe Symbole
\fi

%-------------------------------------
% Spracheinstellungen
%-------------------------------------
\usepackage[ngerman]{babel} % neue deutsche Sprache
\babelprovide[hyphenrules=ngerman-x-latest]{ngerman} % neuste Trennregeln

%-------------------------------------
% Satzspiegel (Seitenlayout)
%-------------------------------------
% Der Satzspiegel wird bei Komascript Klassen basierend auf den Angaben DIV und BCOR automatisch berechnet. Möchte man die Ränder selber angeben, kann dies mit dem Paket geometry geschehen.
%\usepackage[left=3cm, right=3cm, top=2.5cm, bottom=2.5cm]{geometry}
\usepackage{geometry}
% Falls 1,5facher Zeilenabstand gefordert ist, die nächsten zwei Zeilen auskommentieren
%\usepackage{setspace}
%\onehalfspacing

% --------------------------------------
% Standdard Pakete
% --------------------------------------
\usepackage{blindtext} % Blindtext erzeugen
\usepackage{microtype} % bessere Typographie
\usepackage{amsmath} % Mathematik
\usepackage{siunitx} % Einheiten
\usepackage{graphicx} % Bilder einbinden
\graphicspath{{figs/}} % Ordner für Bilddateien
\usepackage{wrapfig} % Bilder im Text
\usepackage{float} % Positionierung von Abb. und Tab. mit [H] erzwingen
\usepackage{booktabs} % schönere Tabellen
\usepackage{tabularray} % Tabellen leichter gestalten
\UseTblrLibrary{booktabs, siunitx}
\usepackage{subcaption} % Für 1a, 1b, 1c Abbildungen

%-------------------------------------------------
% Zusätzliche Pakete, gerne erweitern
%-------------------------------------------------
\usepackage[colorinlistoftodos]{todonotes} % TODO-Liste
\usepackage{dirtree} % Visualieren von Ordnerstrukturen
\usepackage{menukeys}

%-------------------------------------------------
% Quellenverzeichnis konfigurieren
%-------------------------------------------------
\usepackage{csquotes, xpatch} % Empfohlen für die Verwendung von BibLaTex
\usepackage[
	backend=biber, % Muss eingestellt werden!
	% Stil alphabetic
	% andere Möglichkeiten: numeric, authoryear etc.
	style=alphabetic,
	maxcitenames=3, % maximal 3 Autoren, danach et. al.
	doi=false, % keine DOI in Literaturverzeichnis
	isbn=false, % keine ISBN in Literaturverzeichnis
]{biblatex}
\addbibresource{bibs/Literatur.bib}
% Hier weitere .bib Datein hinzufügen

% ----------------------------------------
% Layout von Kopf- und Fußzeile
% ---------------------------------------
\usepackage[automark]{scrlayer-scrpage}
% Aktuelles Kapitel mittig in Kopfzeile
% Seitenzahl mittig in Fußzeile

% Alternatives Layout
% Links den Kolumnentitel und rechts (außen) ein Logo
% Kolumnentitel zeigt aktuelle section ab der ersten section eines Kapitels
%\usepackage[autooneside=false]{scrlayer-scrpage}
%\automark[section]{chapter}
%\clearpairofpagestyles
%\ihead{\headmark} % innen Kolumentitel
%\ohead{\smash{\includegraphics[height=10mm]{THBLogo.pdf}}} 
%\cfoot*{\pagemark} % unten mittig Seitenzahl

% ---------------------------------------
% Einstellungen für Quellcode Datein
% ---------------------------------------
\usepackage{scrhack} % für veraltetes listings package
\usepackage{listings} % Quellcode einbinden
\renewcommand\lstlistlistingname{Quellcodeverzeichnis}

%-----------------------------------
% Umlaute in Code korrekt darstellen
% siehe auch: https://en.wikibooks.org/wiki/LaTeX/Source_Code_Listings
%-----------------------------------
\lstset{literate=
	{á}{{\'a}}1 {é}{{\'e}}1 {í}{{\'i}}1 {ó}{{\'o}}1 {ú}{{\'u}}1
	{Á}{{\'A}}1 {É}{{\'E}}1 {Í}{{\'I}}1 {Ó}{{\'O}}1 {Ú}{{\'U}}1
	{à}{{\`a}}1 {è}{{\`e}}1 {ì}{{\`i}}1 {ò}{{\`o}}1 {ù}{{\`u}}1
	{À}{{\`A}}1 {È}{{\'E}}1 {Ì}{{\`I}}1 {Ò}{{\`O}}1 {Ù}{{\`U}}1
	{ä}{{\"a}}1 {ë}{{\"e}}1 {ï}{{\"i}}1 {ö}{{\"o}}1 {ü}{{\"u}}1
	{Ä}{{\"A}}1 {Ë}{{\"E}}1 {Ï}{{\"I}}1 {Ö}{{\"O}}1 {Ü}{{\"U}}1
	{â}{{\^a}}1 {ê}{{\^e}}1 {î}{{\^i}}1 {ô}{{\^o}}1 {û}{{\^u}}1
	{Â}{{\^A}}1 {Ê}{{\^E}}1 {Î}{{\^I}}1 {Ô}{{\^O}}1 {Û}{{\^U}}1
	{œ}{{\oe}}1 {Œ}{{\OE}}1 {æ}{{\ae}}1 {Æ}{{\AE}}1 {ß}{{\ss}}1
	{ű}{{\H{u}}}1 {Ű}{{\H{U}}}1 {ő}{{\H{o}}}1 {Ő}{{\H{O}}}1
	{ç}{{\c c}}1 {Ç}{{\c C}}1 {ø}{{\o}}1 {å}{{\r a}}1 {Å}{{\r A}}1
	{€}{{\EUR}}1 {£}{{\pounds}}1 {„}{{\glqq{}}}1
}

% ----------------------------------------------------
% Einstellung für Links, Lesezeichen und PDF-Metadaten
% ----------------------------------------------------
% \vref für automatisches Ergänzen von "auf der nächsten (auf dieser) Seite" bei Referenzen.
\usepackage{varioref}
% Farbige links, sowie PDF-Lesezeichen
% Empfohlen für die PDF Version
% Für Print Version entfernen
\usepackage[
	colorlinks=true,
	linkcolor=blue,
	bookmarks=true,
	bookmarksnumbered=true,
	bookmarksopen=true,
	bookmarksopenlevel=0,
	pdfstartpage=1,
]{hyperref}
% Metadaten in PDF eintragen
\hypersetup{
	pdfinfo={
		Title={\myTitel},
		Author={\myAutor},
	}
}
% Ermöglicht "clevere" Referenzen, automatisches Ergänzen des Verweistyps zum Beispiel Abbildung
\usepackage{cleveref}

% ----------------------------------------------------
% Eigene Einstellungen und Befehle
% ----------------------------------------------------
% Hier können weitere Einstellungen vorgenommen werden
% und neue Befehle/Umgebungen definiert werden

\setcapindent{1em} % Einzug für mehrzeilige Beschriftungen
\setlength{\marginparwidth}{2cm} % für todo package
% Befehl für Quellen bei Abbildungen oder Tabellen
\newcommand{\Quelle}[2]{\parbox{#1}{{\small Quelle: #2}}}
\chapter{Latex Beispiele}
\section{Abbildungen}
Mit der \texttt{figure}-Umgebung kann man eine Abbildung erstellen.

% Mögliche Platzierungen
% h	= hier platziert
% t = oben auf einer Seite
% b = unten auf einer Seite
% p = eigene Seite für Gleitumgebungen
\begin{figure}[ht]
	\centering % Abbildung zentriert
	\includegraphics[width=0.6\linewidth]{THBLogo.pdf} % Bild laden
	\Quelle{0.6\linewidth}{\cite{THBWebsite}} % eigener Befehl für Quellen
	\caption{Das Logo der Technischen Hochschule Brandenburg} % Beschriftung
	\label{fig:THBLogo} % Label für Verweis mit \ref
\end{figure}

In Abbildung \ref{fig:THBLogo} ist das Logo der THB zu sehen.

\section{Eine Tabelle}
Eine Tabelle wird mit der \texttt{table}-Umgebung erstellt. Diese ist aber nur für Platzierung und Überschrift zuständig. Der Inhalt kann mit der \texttt{tabular}-Umgebung oder noch besser mit der \texttt{tblr}-Umgebung gesetzt werden.

\begin{table}[ht]
	\centering
	\caption{Zahlen der THB}
	\label{tab:Beispieltabelle}
	\begin{tblr}{lr}
		\toprule
		Beschreibung & Anzahl \\ \midrule
		Studierende & 2756 \\
		Professoren & 64 \\ 
		Studiengänge & 23 \\
		Fachbereiche & 3 \\ \bottomrule
	\end{tblr}
\end{table}

In Tabelle \ref{tab:Beispieltabelle} stehen Zahlen über die THB.

\section{Listen}
Für Aufzählungen etc.

\subsection{Nummerierte Listen}
\begin{enumerate}
	\item Erstes Element
	\item Zweites Element
	\item usw.
\end{enumerate}

\subsection{Stichpunkte}
\begin{itemize}
	\item Ein Stichpunkt
	\item Noch ein Stichpunkt
	\item usw.
\end{itemize}

\subsection{Liste mit Beschreibung}
\begin{description}
	\item[Fachbereich Technik] Ganz was Feines
	\item[Fachbereich Informatik] Auch ok
	\item[Fachbereich Wirtschaft] \blindtext
\end{description}
